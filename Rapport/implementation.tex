\chapter{Implementation}
Following is a list of the resources used in this system implementation:
\begin{itemize}
\item Python for implementation of the system
\item Python framework Twisted for network communication
\item Website?
\end{itemize}

\section{Python}
Python \cite{python} was chosen as the programming language for the system implementation in this thesis. This is due to Pythons multipurpose abilities, and the intuitive syntax and programming style, which enables a rapid program development. Python also possesses characteristics like object-orientation, as well as being modular and dynamic. There exists a wide range of modules and libraries in Python. One of these libraries is the unit test framework, which supports test automation and and eases code testing.

\section{Twisted}
Twisted \cite{twisted} is a networking engine written in Python.
Some of the main components of the Twisted library are described below:
\begin{itemize}
\item \textbf{Reactor}: The reactor reacts to events in a loop and dispatches them to predetermined callback functions that handle the events. The event loop runs endlessly, unless it is told to stop. 

\item \textbf{Protocols}: Each protocol object in Twisted represents one connection. Protocols handle network events in an asynchronously manner. The protocol is responsible for handling incoming data, and new connections and lost connections of peers.

\item \textbf{Factory}: Protocol instances are created in the factory, one for each connection. The factory utilize the protocol for communication with its peers. Information that is persistent across connections is stored in the factory.

\item \textbf{Transport}: A transport is a method that represents the actual connection between the two endpoints in a protocol, e.g. a TCP connection. The transport is used for communication between the two endpoints, as it writes data from one connection to the other.
\end{itemize}
