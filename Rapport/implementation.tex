\chapter{Implementation}
This chapter describes the implementation of the system in this thesis. The first section details the software used in the implementation. The rest of the chapter describes how the individual modules of the system are implemented.
%realisering av selve modulene i implemtasjon


\section{Software}
Following is a list of the resources used in this system implementation:
\begin{itemize}
\item Python for implementation of the system
\item Python framework Twisted for network communication
\item Flask for web applications
\end{itemize}

\subsection{Python}
Python \cite{python} was chosen as the programming language for the system implementation in this thesis. This is due to Pythons multipurpose abilities, and the intuitive syntax and programming style, which enables a rapid program development. Python also possesses characteristics like object-orientation, as well as being modular and dynamic. There exists a wide range of modules and libraries in Python. One of these libraries is the unit test framework, which supports test automation and eases code testing.

\subsection{Twisted}
Twisted \cite{twisted} is a networking engine written in Python.
Some of the main components of the Twisted library are described below:
\begin{itemize}
\item \textbf{Reactor}: The reactor reacts to events in a loop and dispatches them to predetermined callback functions that handle the events. The event loop runs endlessly, unless it is told to stop. 

\item \textbf{Protocols}: Each protocol object in Twisted represents one connection. Protocols handle network events in an asynchronously manner. The protocol is responsible for handling incoming data, and new connections and lost connections of peers.

\item \textbf{Factory}: Protocol instances are created in the factory, one for each connection. The factory utilizes the protocol for communication with its peers. Information that is persistent across connections is stored in the factory.

\item \textbf{Transport}: A transport is a method that represents the actual connection between the two endpoints in a protocol, e.g. a TCP connection. The transport is used for communication between the two endpoints, as it writes data from one connection to the other.
\end{itemize}

\subsection{Flask}

\section{Blockchain Modules}
This section contains descriptions of the modules implemented in the blockchain layer.

\subsection{Block Module}
The block module consists of a \textit{Block} class. The attributes of the class are the index of the block, which is the number of the block in the blockchain where the genesis block starts at 0; the hash of the block immediately preceding the block; a timestamp of when the block was created; the transactions included in the block; and the new hash of the block. 

The block hash is calculated using the python hashlib \cite{hashlib} library. It uses SHA-256 hash algorithm based on the index, previous hash, timestamp and transactions in the block, thus creating a unique hash for every block. The \textit{hexdigest} method is used to return a string object of double size, containing only hexadecimal digits, for better readability. 

Other methods in the class include \textit{validate\textunderscore block} for validation of a new block, based on the previous block: \textit{propose\textunderscore block} to create a new block, based on the previous block and the current time; and \textit{assert\textunderscore equal} to verify that two blocks are identical. 


\subsection{Network Module}
A peer-to-peer(p2p) network is implemented in the network module, using the Twisted \cite{twisted} framework. 

The initial network node starts a server, and a client that connects to the server. Other nodes also start a server on a specified port number and a client which connects to a server already running on a given IP address and port number. 

The module consists of two classes, a \textit{PeerManager} and a \textit{Peer}. The \textit{PeerManager} class is a Twisted factory, and is responsible for storing information about the peer, which is persistent between connections. This includes attributes such as a dictionary containing all connections to other peers, methods for adding and removing peers to the dictionary, as well as a method for starting a new client that connects to a new peer's server.

\textit{Peer} is a subclass of the Twisted protocol \textit{IntNStringReceiver}. This means that each received message is a callback to the method \textit{stringReceived}. The \textit{Peer} class also keeps track of information in a connection between two peers. This includes a method for discovering when the connection is lost. The main method of the \textit{Peer} class is the \textit{stringReceived} method. Based on what message type was received, the method decides what to do with the message. 

The initial message sent by a client, node A, connecting to a new server, node B, is the \textit{hello} message. This includes the client's node id, IP address, and host port which is information the server on node B keeps track of for all its peers. If the connection is successful, the server on node B acknowledges the message by sending its own node id, IP address, and port number for node A to store. Server B proceeds by sending a message containing information about all its peers to node A. Node A then starts a new client for all the peers it is not already connected to and repeats the process described above. 

Other messages received are processed in the factory, and further handled by the \textit{Node} object.


\subsection{Node Module}
%Make float chart
The node module consists of a \textit{Node} class, which is a subclass of the \textit{PeerManager} class. The main component of the \textit{Node} class is the state machine, which is triggered by the reactor making a \textit{LoopingCall} at periodic intervals. 

Every time the \textit{state\textunderscore machine} method is executed, the network leader sends out an \textit{append entries} RPC to all its followers. This lets the followers know that the leader is still operating. If new transactions are available, the leader creates a new block including these transactions, and proposes the block to its followers. If a follower finds a block valid, its stores the block in its log of proposed blocks during the next execution of the \textit{state machine} method. 

The leader keeps a timer on the proposed block. If a timeout occurs before a majority has validated and accepted the block, the leader steps down, and a new election for a leader will start once the \textit{leader election timeout} occurs. However, if the majority validates and accepts the block before the timeout occurs, the leader will add the block to the blockchain, and notify its followers to do the same. 

\subsection{Consensus Module}
The consensus module used in this system is Raft. 

The consensus module consists of a \textit{Validator} class. A \textit{Validator} object is created by the \textit{Node} object, to handle the validation and consensus in the blockchain. Once a \textit{Validator} object is created, a \textit{leader election timer} starts. The timeout is cancelled if the node receives a message from a leader. If the timeout occurs, the \textit{start leader election} method is called. A node promotes itself to a candidate, starts a new term and votes for itself as leader before requesting votes from its peers with the \textit{request vote} RPC. 

A follower receiving a \textit{request vote} RPC will vote for the candidate if it has not already vote in that term or not voted for someone else. The follower will also verify that the candidate's blockchain log is at least as up to date as its own before casting the vote. If the candidate receives a majority vote it will establish itself as leader by sending out an \textit{append entries} RPC. 

Another possible outcome of the election process is that the candidate does not receive a majority vote before the election timeout occurs. It will then either promote itself as a candidate and start a new term, or receive a \textit{request vote} RPC from another candidate, who has started a new term.

The third outcome is that several nodes become candidate at the same time. If a candidate receives an \textit{append entries} RPC from a different node, it will step down and become a follower. 

As previously mentioned, leaders send out \textit{append entries} RPCs from the \textit{Node} object. Followers respond to the RPC in the \textit{Validator} object. With every RPC, the leader includes the previous log index and log term. In the \textit{respond append entries} method, a follower checks if it has this entry in its log and responds accordingly. If there is no entry, the leader decreases the index until a match is found. If there is a conflicting entry, the follower deletes its own entry and writes the leaders entry in the log. 

If the \textit{append entries} RPC includes a new proposed block, the follower will validate the block and write the block to the log during the next run of the \textit{state machine}. 


\section{Application Layer}
This section includes descriptions of modules implemented in the application layer of the system. 

\subsection{Settlement} \label{settlement}
Settlement occurs based on existing smart contracts in the system, and is triggered when the parties of the contract have produced and/or consumed electricity. 

Three assumptions were made for the settlement process in the system.
\begin{itemize}
\item During any interval, the amount of electricity consumed in the system equals the amount of electricity produced in the system.
\item If the consumer of a contract has consumed more electricity than the producer of the contract has produced, the excess consumed electricity is produced by the pure producers in the system.
\item If a prosumer produces more electricity than is consumed by itself or the consumer in the contract, during an interval, it is assumed that the electricity is stored in batteries and not part of the settlement process.
\end{itemize}

The settlement process is accumulative and based on previous data generated in the system. Each node has a \textit{account} where settlement info is stored. 

\subsection{Smart Contracts}
Smart contracts enable automatic transfer of assets, once conditions are met. Thus, in contrast to traditional contracts, smart contracts do not only define the terms and conditions of an agreement, they also provide a method for enforcing the agreement.
%Turing complete

\subsection{User Interface}
The user interface was not integrated with the rest of the system, rather a interface layout was outlined as seen from the consumers perspective. The interface consisted of a web page with two menu options: one for monitoring consumed and produced electricity in real-time; and one to create smart contracts based on prosier availability.
