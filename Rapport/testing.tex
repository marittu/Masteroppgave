\chapter{Testing}\label{test}
This chapter details how the system was tested. Section \ref{system} describes how the complete system was tested, while section \ref{moduletest} specifies how the individual modules of the system were tested.

\section{System Test} \label{system}
A functional test was run on the complete system, using pseudo-random simulation data. The test approach is summarized below:
\begin{itemize}
\item Four prosumer nodes with 48 hours worth of test data for produced and consumed electricity, given in kWh per minute. Each node periodically read the test data from a CSV (Comma Separated Value) file. The four nodes were run as different processes on the same computer. 
\item Smart contracts were manually constructed in pairs between the node, with one node selling, and one node buying electricity per contract.
\item Transactions were settled according to the settlement module described in section \ref{settlement}. For simplicity, the price of 1 kWh was set to 1 NOK, regardless of the production source.
\item All transactions were stored in the blockchain. Transactions from all nodes during a given interval were stored in the same block. 
\end{itemize}

\section{Module Testing} \label{moduletest}
\subsection{User Interface}

Tests were executed on the implemented functionality in the user interface. 
The monitoring of consumed and produced electricity of a node was tested using simulated data stored in a CSV file. 

For the smart contract API, a simple table of producer availability was constructed using arbitrary data, and a form for consumers to initiate a contract was created.  
 
\subsection{Unittests}
Several modules were tested with unittests. Even though the correct behavior of the settlement function can be verified from the system test, a unittest was also created to ease the maintenance process, as it provides a fast and accurate way to verify the correctness of potential changes in the module. The block module was also tested with a unittest for the same reasons.

\subsection{Network Module}
The ability for nodes to communicate over the network is crucial for the system to function. Several aspects of this module were tested:
\begin{itemize}
\item Communication with two machines on the same network.
\item Communication with two machines on two separate networks.
\end{itemize}

\subsection{Consensus}
The consensus module is an important feature in order for the system to function with the correct behavior. The following tests were performed to verify this.
\begin{itemize}
\item Ability to correct wrong data on a node.
\item During normal operation, the consensus model was tested on 10 nodes running as separate processes on two separate computers.
\item Adding new nodes under operation.
\item Election of new leader after a leader has crashed.
\end{itemize}

\subsection{Smart Contract}
Signing and validating 



