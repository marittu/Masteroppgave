\chapter{Testing}
Test with at least 10 nodes
Ideally test over Ethernet/several computers
\section{Unittest}