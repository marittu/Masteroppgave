\chapter{Conclusion}
To build such a system on top of existing blockchains such as Bitcoin or Ethereum, may result in a costly system, due to relatively large transactions fees and many, low-value transactions in the microgrid network. Ethereum, in addition, requires Ether to run applications/smart contracts on the EVM. Using the modular blockchain of Hyperledger might be a better idea, however - some cost?

The blockchain implemented in this thesis is able to run and process all transactions in the network at no additional cost.

Consortium - node ids are linked to identities - if a node tries to attack the blockchain, its identity will be known and actions may be taken thereafter. 

Transparency - account owners can at any time review there account actions. Super user can see everything

Problem - all users can review all other users account activity through the blockchain ledger - however it is a rigorous task to read all account activity from the ledger. Thus, the application layer exist to make it easier to monitor account activity - and this is only available to the account owner. 

Blockchain is still a fairly new technology with new advances at a rapid speed. They have already shown great potential to work in many different areas, including in an energy system, as showed in this thesis. Extensive testing and evaluation of possible implementation should be executed before it can run in the proposed system.
%In general, this section should provide the final "take home message" from your work. What can other researchers in your field learn from your work?