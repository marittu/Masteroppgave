\chapter*{Problem Description}
In a limited electrical system (a microgrid, such as an island community) there will be some, one or more, energy producers (e.g. diesel generators), some consumers and some, zero or more, prosumers (consumers who also produce energy, e.g. by solar panel or wind turbines). In addition, there may be energy storage such as batteries.

The prosumer may at times produce more energy than they themselves use at the moment and sell this surplus energy to a neighbor who needs it. In an advanced solution, one can imagine that the prosumers have a separate battery that they can choose to charge instead of selling the energy. There are three sources of energy for the consumer in the described limited energy system; the pure manufacturers (e.g. diesel engines), the batteries (which sometimes buy energy) and the prosumers.

In the system, there are also consumers, who buy energy. The price of delivered energy may vary from source to source and over time. An optimization of energy based on different criteria may be interesting, but this will not be considered. In order to settle the value of the energy flowing in the system, one wants to look at the use of blockchain technology.

The task will be:

\begin{itemize}
\item Look at how blockchain technology is used and can be used in microgrids.
\item Suggest a blockchain-like system that can calculate the value of the energy flowing. This system may have a user interface where offer and demand are displayed.
\item Implement the proposed system and test it using simulated data.
\end{itemize}